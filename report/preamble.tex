%\usepackage[danish]{babel}
\usepackage[utf8]{inputenc}
\usepackage{fancyhdr}
\usepackage{lipsum}
\usepackage{import}
\usepackage{graphicx}
\usepackage{pdfpages}
\usepackage{wrapfig}
\usepackage{markdown}
\usepackage{tabularx}
% biblagorafi stuff
\usepackage[backend=biber, style=alphabetic, citestyle=authoryear]{biblatex}

\usepackage[T1]{fontenc} %thanks's daleif
\usepackage[utf8]{inputenc}
%\usepackage[english, danish]{babel}
\usepackage{pdfpages}

% bruges til grafikken på forsiden
%\usepackage[printwatermark]{xwatermark}
\usepackage{xcolor}
%\usepackage[firstpage]{draftwatermark}
\usepackage{transparent}


%Nedenfor er alle pakkerne der bruges. Det er nok løgn - de fleste bruges nok ikke til noget, => DO NOT TOUCH!!!
\usepackage[utf8]{inputenc}
\usepackage{pdfpages}
\usepackage{rotating}
\usepackage{graphicx}
\usepackage{amsmath}
\usepackage{titlesec}
\setcounter{secnumdepth}{4}
\usepackage{subfiles}
\usepackage[T1]{fontenc}
\usepackage{float}
\usepackage{fancyvrb}
\usepackage{color}
\usepackage{hyperref}
\hypersetup{
    colorlinks=true,
    linkcolor=blue,
    filecolor=magenta,      
    urlcolor=cyan,
}

\usepackage{tikz}
\usetikzlibrary{positioning}
\usetikzlibrary{arrows,automata}
\usepackage{rotating}
\usepackage{caption}
\usepackage{amssymb}
\usepackage{listings}
\usepackage{pdfpages}
\usepackage{mathtools}
\usepackage[version=4]{mhchem}
\usepackage{makecell}
\usepackage{longtable}

\usepackage{siunitx}
\sisetup{range-phrase= - }
%-----
\usepackage{multirow}
\usepackage{booktabs}
\usepackage{subcaption}
\usepackage{framed}
% for \begin{comment}:
\usepackage{verbatim}

\setlength{\parindent}{0pt}
\setlength{\parskip}{1ex plus 0.5ex minus 0.2ex}
\usepackage{multicol}
\usepackage{slantsc}
\usepackage{lmodern}
\usepackage{wrapfig}
\usepackage[nottoc,numbib]{tocbibind}
\usepackage[normalem]{ulem}
\useunder{\uline}{\ul}{}
\usepackage{tabularx}

% code thingy
\lstset{ 
  backgroundcolor=\color{white},   % choose the background color; you must add \usepackage{color} or \usepackage{xcolor}; should come as last argument
  basicstyle=\small \ttfamily,      % the size of the fonts that are used for the code
  breakatwhitespace=false,         % sets if automatic breaks should only happen at whitespace
  breaklines=true,                 % sets automatic line breaking
  captionpos=b,                    % sets the caption-position to bottom
  commentstyle=\color{green},      % comment style
  deletekeywords={...},            % if you want to delete keywords from the given language
  escapeinside={\%*}{*)},          % if you want to add LaTeX within your code
  extendedchars=true,              % lets you use non-ASCII characters; for 8-bits encodings only, does not work with UTF-8
  %firstnumber=1,                   % start line enumeration with line 1000
  frame=single,	                   % adds a frame around the code
  keepspaces=true,                 % keeps spaces in text, useful for keeping indentation of code (possibly needs columns=flexible)
  keywordstyle=\color{blue},       % keyword style
  stringstyle=\color{red},         % set string color
  language=Java,                   % the language of the code
  morekeywords={*,...},            % if you want to add more keywords to the set
  numbers=left,                    % where to put the line-numbers; possible values are (none, left, right)
  numbersep=5pt,                   % how far the line-numbers are from the code
  numberstyle=\tiny\color{mygray}, % the style that is used for the line-numbers
  rulecolor=\color{black},         % if not set, the frame-color may be changed on line-breaks within not-black text (e.g. comments (green here))
  showspaces=false,                % show spaces everywhere adding particular underscores; it overrides 'showstringspaces'
  showstringspaces=false,          % underline spaces within strings only
  showtabs=false,                  % show tabs within strings adding particular underscores
  stepnumber=2,                    % the step between two line-numbers. If it's 1, each line will be numbered
  stringstyle=\color{mymauve},     % string literal style
  tabsize=2,	                   % sets default tabsize to 2 spaces
  %title=\lstname                   % show the filename of files included with \lstinputlisting; also try caption instead of title
}